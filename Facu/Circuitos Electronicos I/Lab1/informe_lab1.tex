%%%%%%%%%%%%%%%%%%%%%%%%%%%%%%%%%%%%%%%%%%%%%%%%%%%%%%%%%%%%%%%%%%%%%%%%%%%%%%%%
%2345678901234567890123456789012345678901234567890123456789012345678901234567890
%        1         2         3         4         5         6         7         8
\documentclass[letterpaper, 10 pt, conference]{ieeeconf}  % Comment this line out
                                                          % if you need a4paper
%\documentclass[a4paper, 10pt, conference]{ieeeconf}      % Use this line for a4

\usepackage{float}
                                                          % paper
% uso paquete bookmark para tener bien los outlines.
\usepackage{bookmark}

% Configuro el idioma.
\usepackage[utf8]{inputenc} % Importante para mantener acentos.
\usepackage[spanish, activeacute]{babel} % Requiere: texlive-lang-spanish. Por primera vez hay que ejecutar: texconfig init> log

% Paquete para poder usar acentos en $$.
\usepackage{mathtools}
%\setmathfont{XITS math}

% package to get \url
\usepackage{hyperref}
\hypersetup{
  colorlinks=true,
  linkcolor=magenta,
  filecolor=magenta,
  citecolor=magenta,      
  urlcolor=magenta,
}

% Graficos electrónicos
\usepackage{circuitikz}

\IEEEoverridecommandlockouts                              % This command is only
                                                          % needed if you want to
                                                          % use the \thanks command
\overrideIEEEmargins
% See the \addtolength command later in the file to balance the column lengths
% on the last page of the document

\usepackage{graphicx}
\usepackage{graphics}

% styling for matlab/octave code.
\usepackage{matlab-prettifier}
% Configuracion, con esto puede agregar ñ.
\lstset{
  literate={ñ}{{\~n}}1
}

% The following packages can be found on http:\\www.ctan.org
%\usepackage{graphics} % for pdf, bitmapped graphics files
%\usepackage{epsfig} % for postscript graphics files
%\usepackage{mathptmx} % assumes new font selection scheme installed
%\usepackage{times} % assumes new font selection scheme installed
\usepackage{amsmath} % assumes amsmath package installed
%\usepackage{amssymb}  % assumes amsmath package installed

\title{\LARGE \bf Laboratorio N° 1}

\author{
  Tom\'as Vidal\\
  {\it Circuitos Electrónicos 1}\\
  {\it Facultad de Ingenier\'ia, UNLP, La Plata, Argentina.}\\
  {\it 26 de Marzo, 2024.}
}                                            % <-this % stops a space


% comienzo

% INTRO


% Figura
\newcommand{\image}[2] {
  \begin{figure}[H]
    \centering
    \includegraphics[width=0.43\textwidth]{../figures/#1.png}
    \caption{#2}
    \label{fig:#1}
  \end{figure}
}

% Codigo
% \begin{lstlisting}[style=Matlab-editor]
% % el código va aca
% dispc("HELLO WORLD");
% \end{lstlisting}

\begin{document}
\maketitle
\thispagestyle{empty}
\pagestyle{empty}

\section{INTRODUCCCI\'ON}
A continuación se detallan los procedimientos y resultados del laboratorio de Circuitos Electrónicos 1. Se emplearon dos configuraciones para un amplificador operacional (inversora y no inversora), se efecturaron análisis de frecuencias y ganancias para diferentes valores de los componentes involucrados, los mismos se especifican en los resultados. Se concluye que el producto de ganancia ancho de banda se conserva para una misma topología. Posteriormente se realizaron pruebas sobre una placa de ensayo en la que aparecían no linealidades, y luego de activar la realimentación desde la salida se solucionó la no linealidad.


\section{Circuito a trabajar}
\begin{circuitikz}
    % Op-Amp
    \draw (2,0) node[op amp] (opamp) {};
    
    % Inputs
    \draw (opamp.-) to ++(-1,0) node[left] {$V_{in}$};
    \draw (opamp.+) to ++(-1,0) node[left] {$V_{in}^+$};
    
    % Feedback resistor
    \draw (opamp.out) -- ++(1,0) to [R=$R_f$] ++(0,-2) node[ground]{};
    
    % Input resistor
    \draw (opamp.-) -- ++(0,1) to [R=$R_1$] ++(-2,0) node[ground]{};
    
    % Output
    \draw (opamp.out) -- ++(1,0) node[right] {$V_{out}$};
\end{circuitikz}

% \subsection{Obtenci\'on de la ecuaci\'on del sistema}
% Analizando el diagrama en bloques provisto se obtiene la siguiente ecuaci\'on que describe su comportamiento.
% \begin{equation} \label{eq:y}
%   y[n] = \frac{1}{2}x[n] + \frac{1}{2}x[n-1] + R^{L}y[n-L]
% \end{equation}
% Se obtiene la transferencia del sistema transformando ec. \ref{eq:y} con la transformanda $Z$.
% \begin{equation} \label{eq:Hz}
%   H(z) = \frac{Y(z)}{X(z)} = (\frac{1}{2})\frac{1 + z^{-1}}{1 - R^{L}z^{-L}}
% \end{equation}
% 
% Mediante la funci\'on de transferencia se obtiene la respuesta en frecuencia (fig. \ref{fig:HfreqResp}) del sistema de la siguiente manera
% \begin{equation*}
%   H(e^{j2{\pi}s}) = (\frac{1}{2})\frac{1 + {e^{-j2{\pi}s}}}{1 - R^{L}{e^{-j2{\pi}sL}}}
% \end{equation*}
% % TODO hacer una animación para diferentes valores de L y R
% \image{HfreqResp}{Respuesta en frecuencia del sistema (ec. \ref{eq:y})}
% 
% \subsection{Polos y ceros de H(z)}
% Se grafican los polos y los ceros de la funci\'on de transferencia (ec. \ref{eq:Hz}) con el siguiente c\'odigo
% % Codigo
% % TODO: actualizar esto con el correspondiente codigo
% \begin{lstlisting}[style=Matlab-editor]
% % tf y pzmap son del paquete de control
% pkg load control;
% % Se define el tipo de z
% z = tf('z', Ts);
% % Se define la funcion de transferencia H(z) como H
% H = @(r, l) (1/2)*( (1+z^(-1)) / (1-(r^l)*(z^(-l))) );
% % Se define una secuencia aleatoria uniforme de longitud 100
% x = rand(100);
% L = length(x);
% R = .1;
% % Se grafican los zeros y polos de H
% pzmap(H(R, L));
% \end{lstlisting}
% 
% \image{Hz-pzmap-1}{Polos y ceros de la función de transferencia (ec. \ref{eq:Hz}), con R=10}
% \image{Hz-pzmap-2}{Polos y ceros de la función de transferencia (ec. \ref{eq:Hz}), con R=5}
% \image{Hz-pzmap-3}{Polos y ceros de la función de transferencia (ec. \ref{eq:Hz}), con R=1}
% \image{Hz-pzmap-4}{Polos y ceros de la función de transferencia (ec. \ref{eq:Hz}), con R=0,5}
% \image{Hz-pzmap-5}{Polos y ceros de la función de transferencia (ec. \ref{eq:Hz}), con R=0,1}
% \image{Hz-pzmap-6}{Polos y ceros de la función de transferencia (ec. \ref{eq:Hz}), con R=0,01}
% 
% Haciendo un an\'alisis de la ec. \ref{eq:Hz} podemos observar que el numerador tiende a cero cuando $z$ tiende a $-1$, y el denominador cuando $z$ tiende a $R$; por lo que los polos se encuentran en la circunferencia de radio $R$ y son L (en cantidad), adem\'as tiene un cero en $z=-1$, que es acorde con lo que se ver en los gr\'aficos anteriores.
% 
% \subsection{Estabilidad del sistema}
% Para que el sistema sea estable y causal todos los polos se deben encontrar dentro de la circunferencia de radio unitario. Por lo que para valores de $0<R<1$ el sistema es estable.
% 
% \section{Inciso 2}
% \subsection{Filtro de pasatodo}
% Obtengo la ecuación (\ref{eq:yfiltro}) que describe el sistema a partir de los diagramas en bloques dados.
% \begin{equation} \label{eq:yfiltro}
%   y[n] = a.x[n] + x[n-1] - a.y[n-1]
% \end{equation}
% Calculo la función de transferencia a partir de la ecuación \ref{eq:yfiltro}.
% \begin{equation} \label{eq:Hfiltro}
%   H(z) = \frac{a + z^{-1}}{1+a.z^{-1}}
% \end{equation}
% \subsection{Fase del filtro}
% En el siguiente se muestran diferentes fases de la respuesta en frecuencia del filtro (ec \ref{eq:yfiltro}).
% \image{HPhiA}{Diferentes fases de la respuesta en frecuencia del filtro (ec. \ref{eq:yfiltro}).}

\begin{thebibliography}{99}

% TODO
\bibitem{tftd_tp5}ANÁLISIS DE SISTEMAS Y SE\~{n}ALES - A\~{n}O 2022, Práctica 5 Transformada de Fourier de Tiempo Discreto (TFTD), Serie Discreta Fourier (SDF).
\bibitem{tftd_teoria}ANÁLISIS DE SISTEMAS Y SE\~{n}ALES - A\~{n}O 2022, Filminas de teor\'ia 5: Transformada de Fourier de Tiempo Discreto (TFTD).

\end{thebibliography}

\end{document}
